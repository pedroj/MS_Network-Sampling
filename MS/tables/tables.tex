\documentclass[11pt]{amsart}
\usepackage{geometry}                % See geometry.pdf to learn the layout options. There are lots.
\geometry{letterpaper}                   % ... or a4paper or a5paper or ... 
%\geometry{landscape}                % Activate for for rotated page geometry
%\usepackage[parfill]{parskip}    % Activate to begin paragraphs with an empty line rather than an indent
\usepackage{graphicx}
\usepackage{amssymb}
\usepackage{epstopdf}
\DeclareGraphicsRule{.tif}{png}{.png}{`convert #1 `dirname #1`/`basename #1 .tif`.png}

\title{Tables}
\author{Pedro Jordano}
%\date{}                                           % Activate to display a given date or no date

\begin{document}
\maketitle
%%%%%%%%%%%%%%%%%%%%%%%%%%%%%%%%%%%%%%%%%%%%%%%%%%%%%%%%%%%%%%%%%%%%%%%%%%%%%%%%%%
\section*{Table captions}
\noindent \textbf{Table~1.} A taxonomy of link types for ecological interactions (Olesen et al. 2011). $A$, number of animal species; $P$, number of plant species; $I$, number of observed links; $C= 100 I/(AP)$, connectance; $FL$, number of forbidden links; and $ML$, number of missing links. As natural scientists, our ultimate goal is to eliminate $ML$ from the equation $FL = AP - I - ML$, which probably is not feasible given logistic sampling limitations. When we, during our study, estimate $ML$ to be negligible, we cease observing and estimate $I$ and $FL$.\\

\noindent \textbf{Table~2.} Frequencies of different type of forbidden links in natural plant-animal interaction assemblages. $AP$, maximum potential links, $I_{max}$; $I$, number of observed links; $UL$, number of unobserved links; $FL$, number of forbidden links; $FL_P$, phenology; $FL_S$, size restrictions; $FL_A$, accessibility; $FL_O$, other types of restrictions; $ML$, unknown causes (missing links). Relative frequencies (in parentheses) calculated over $I_{max}= AP$ for $I$, $ML$, and $FL$; for all forbidden links types, calculated over $FL$. References, from left to right: Olesen et al. 2008; Olesen \& Myrthue unpubl.; Snow \& Snow 1972 and Jordano et al. 2006; Vizentin-Bugoni et al. 2014; Jordano et al. 2009; Olesen et al. 2011.  \\

\noindent \textbf{Table~3.} A vectorized interaction matrix.\\

\noindent \textbf{Table~4.} Sampling statistics for three plant-animal interaction networks (Olesen et al. 2011). Symbols as in Table 1; $N$, number of records; $Chao1$ and $ACE$ are asymptotic estimators for the number of distinct pairwise interactions $I$ (Hortal et al. 2006), and their standard errors; $C$, sample coverage for rare interactions (Chao \& Jost 2012). Scaled asymptotic estimators and their confidence intervals ($CI$) were calculated by weighting $Chao1$ and $ACE$ with the observed frequencies of forbidden links. \\
%
\newpage

\begin{table}[h!]
  \caption{}
  \begin{center}
		\begin{tabular}{lcl}
      \hline
\\Link type   &  Formulation   &   Definition\\\\
      \hline
\\Potential links&$I_{max}= AP$&Size of network matrix, i.e. maximum number of \\ && potentially observable interactions;  \\ && $A$ and $P$, numbers of interacting animal and \\ &&  plant species, respectively.\\\\
Observed links&$I$&Total number of observed links in the network given \\ &&  a sufficient sampling effort. Number of ones in the  \\ && adjacency matrix.\\\\
Unobserved links&$UL= I_{max} - I$&Number of zeroes in the adjacency matrix.\\\\
Forbidden links&$FL$&Number of links, which remain unobserved because  \\ && of linkage constraints, irrespectively of sufficient  \\ && sampling effort.\\\\
Missing links&$ML= AP - I - FL$&Number of links, which may exist in nature but need  \\ && more sampling effort and\slash or additional sampling  \\ && methods to be observed.\\\\
      \hline
		\end{tabular}
 	\end{center}
\end{table}
%
\newpage
\begin{table}[h!]
  \caption{}
  \label{Table_2}
  \begin{center}
    \begin{tabular}{lcccccc}
      \hline
\\&& {Pollination} &&& {Seed dispersal}&\\
Link type&Zackenberg&Grundvad&Arima Valley&Sta. Virginia&Hato Rat\'on&Nava Correhuelas\\\\
      \hline
\\$I_{max}$&1891&646&522&423&272&825\\
$I$&268 (0.1417)&212 (0.3282)&185 (0.3544)& 86 (0.1042)&151 (0.4719)&181 (0.2194)\\
$UL$&1507 (0.7969)&434 (0.6718)&337 (0.6456)& 337 (0.4085)&169 (0.5281)&644 (0.7806)\\
$FL$&530 (0.3517)&107 (0.2465)&218 (0.6469)& 260 (0.7715)&118 (0.6982)&302 (0.4689)\\
$FL_P$&530 (1.0000)&94 (0.2166)&0 (0.0000)& 120 (0.1624)&67 (0.3964)&195 (0.3028)\\
$FL_S$& $\cdots(\cdots)$&8 (0.0184)&30 (0.0890)& 140 (0.1894)&31 (0.1834)&46 (0.0714)\\
$FL_A$&$\cdots(\cdots)$&5 (0.0115)&150 (0.445)$^a$&$\cdots(\cdots)$&20 (0.1183)&61 (0.0947)\\
$FL_O$&$\cdots(\cdots)$&$\cdots(\cdots)$&38 (0.1128)$^b$&$\cdots(\cdots)$&$\cdots(\cdots)$&363 (0.5637)\\
$ML$&977 (0.6483)&327 (0.7535)&119 (0.3531)& 77 (0.1042)&51 (0.3018)&342 (0.5311)\\\\
      \hline
\\\multicolumn{7}{l}{$^a$, Lack of accessibility due to habitat uncoupling, i.e., canopy-foraging species vs. understory species.}\\
\multicolumn{7}{l}{$^b$, Colour restrictions, and reward per flower too small relative to the size of the bird.}\\\\
      \hline
    \end{tabular}
  \end{center}
\end{table}
%
\newpage
\begin{table}[h!]
  \caption{}
  \label{TabENFA}
  \begin{center}
    \begin{tabular}{lccccc}
      \hline
\\Interaction&Sample\ 1&Sample\ 2&Sample\ 3&{\ldots}&Sample\ \emph{i}\\\\
      \hline
\\A1 - P2&12&2&0&{\ldots}&6\\
A1 - P2&0&0&0&{\ldots}&1\\
{\ldots}&{\ldots}&{\ldots}&{\ldots}&{\ldots}&{\ldots}\\
A5 - P3&5&0&1&{\ldots}&18\\
A5 - P4&1&0&1&{\ldots}&3\\
{\ldots}&{\ldots}&{\ldots}&{\ldots}&{\ldots}&{\ldots}\\
A\textsubscript{i} - P\textsubscript{i}&1&0&1&{\ldots}&2\\\\
      \hline
    \end{tabular}
  \end{center}
\end{table}
%
\newpage
\begin{table}[h!]
  \caption{}
  \label{TabENFA}
  \begin{center}
    \begin{tabular}{lccc}
      \hline
\\       &Hato Rat\'on  &  Nava Correhuelas&    Zackenberg\\\\
      \hline             
\\$A$&17&33&65\\
$P$&16&25&31\\
$I_{max}$&272&825&1891\\
$N$&3340&8378&1245\\
$I$&151&181&268\\
$C$&0.917&0.886&0.707\\
$Chao1$&$263.1\pm70.9$&$231.4\pm14.2$&$509.6\pm54.7$\\
$ACE$&$240.3\pm8.9$&$241.3\pm7.9$&$566.1\pm14.8$\\
$Scaled\ Chao$&195.4&162.7&308.4\\
$CI$&[124.5--266.3]&[148.5--176.9]&[253.6--363.1]\\
$Scaled\ ACE$&178.5&169.7&342.6\\
$CI$&[169.5--187.4]&[161.8--177.6]&[327.8--357.4]\\
$\%\ unobserved^a$&8.33&15.38&47.8\\\\
      \hline
\\\multicolumn{4}{l}{$^a$, estimated with library Jade (R Core Development Team 2010,Chao et al. 2015)}\\
%      \hline
    \end{tabular}
  \end{center}
\end{table}
\end{document}  